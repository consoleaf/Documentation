\documentclass[a4paper,12pt,reqno]{article}

\usepackage{styledoc19}

\begin{document} % конец преамбулы, начало документа
	
	
	\year{2020}
    \docNumber{RU.17701729.04.03-01 ТЗ 01-1-ЛУ}
	\docFormat{Техническое задание}
	\student{БПИ 196}{Якшибаев А.А.}
	\studentt{БПИ 196}{Х. Дилавар А.Ш.Х.}
	
	\project{Информационная система <<Учёт техники ФКН>>}
	
	\supervisor{Инженер отдела технического обеспечения  факультета компьютерных наук}
	{Р. И. Давыдов}
	
	\firstPage
						\newpage
	\secondPage
						\newpage
	\thirdPage
						\newpage
	\section{Введение}
	\subsection{Наименование программы}
	\subsubsection{Наименование программы на русском языке}
	{Информационная система <<Учёт техники ФКН>>}
	\subsubsection{Наименование программы на английском языке}
	{Data system “Equipment record FCS”}
	\subsection{Краткая характеристика области применения}
	
	\newpage
	\section{Основания для разработки}
	\subsection{Документы, на основании которых ведется разработка}
	
	
	Приказ декана факультета компьютерных наук Национального Исследовательского университета <<Высшая школа экономики>> № {\color{red}{TODO}} от {\color{red}{TODO}}.
	
	\subsection{Наименование темы разработки}
	
	Наименование темы разработки – {\color{red}{TODO}}


 Программа выполняется в рамках темы курсовой работы в соответствии с учебным планом подготовки бакалавров по направлению 09.03.04 «Программная инженерия» Национального исследовательского университета «Высшая школа экономики», факультет компьютерных наук.
	
	\newpage 
	\section{Назначение разработки}
	 
	\subsection{Функциональное назначение}
	{\color{red}{TODO}}
	\subsection{Эксплуатационное назначение}
	{\color{red}{TODO}}
	
						\newpage
	\section{Требования к программе}
	\subsection{Требования к функциональным характеристикам}
	\subsubsection{Требования к составу выполняемых функций}
	
	{\color{red}{TODO}}
	\subsubsection{Требования к организации входных данных}
	{\color{red}{TODO}}
	\subsubsection{Требования к организации выходных данных}
	{\color{red}{TODO}}
	%\subsection{Требования к надежности}
	%Для корректной работы программы необходимо иметь 
	\clearpage
	\subsection{Требования к интерфейсу}
	{\color{red}{TODO}}
	\subsection{Условия эксплуатации}
	\subsubsection{Климатические условия}
	Климатические условия сопадают с климатическими условиями эксплуатации\\ устройства \cite{terms}.
	\subsubsection{Требования к пользователю}
	Пользователь должен иметь базовое представление об основных принципах {\color{red}{TODO}}.
	\subsection{Требования к составу и параметру технических средств}
	Для корректной работы приложения необходимо {\color{red}{TODO}}.
	\subsection{Требования к информационной и программной совместимости}
	На устройстве должна быть установлена операционная система {\color{red}{TODO}}.
	\subsection{Требования к маркировке и упаковке}
	Приложение должно быть доступно для скачивания {\color{red}{TODO}}.
	
						\newpage
	\section{Требования к программной документации}
	{\color{red}{TODO}}
	
						\newpage
	\section{Технико-экономические показатели}
	\subsection{Предполагаемая потребность}
	Программа будет использоваться {\color{red}{TODO}}.
	
	\subsection{Ориентировочная экономическая эффективность} 
	{\color{red}{TODO}}.
	
						\newpage
	\section{Стадии и этапы разработки}
	
	\subsection{Необходимые стадии разработки, этапы и содержание работ}
	{\color{red}{TODO}}
	
	% приложения нумеруются отдельно и надо выровнять по правому краю

						\newpage
	\addition{Используемые понятия и определения}
	{\color{red}{TODO}}
						\newpage
	
	\addition{Иллюстрации интерфейса} \label{interface}
	{\color{red}{TODO}}.

						\newpage
	%\section{Источники, использованные при разработке}
	%\renewcommand{\refname}{Список источников}
	% \addcontentsline{toc}{subsection}{\refname}
	\patchcmd{\thebibliography}{\section*{\refname}}{}{}{}
	\addition{Список источников}
	\begin{thebibliography}{3}
		\bibitem{gost}Единая система программной документации – М.: ИПК, Издательство стандартов, 2000, 125 стр.
		\bibitem{lms} 
		LMS [Электронный ресурс] URL: 
		\url{https://lms.hse.ru} (Дата обращения: 05.11.2020, режим доступа: свободный)
		
	\end{thebibliography}

						\newpage
	\listRegistration

\end{document} % конец документа
